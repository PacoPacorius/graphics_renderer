\documentclass{article} 
\usepackage{polyglossia} 
\usepackage{amsmath}
\usepackage{fontspec} 
\usepackage{lipsum} 
\usepackage[margin=1in]{geometry}
\usepackage{graphicx} 
\usepackage{caption} 
\usepackage{subcaption}
\usepackage{hyperref} 
\usepackage{listing}
\hypersetup{% 
    colorlinks=true, linkcolor=blue, filecolor=magenta,      
    urlcolor=cyan, 
    pdfinfo = {%
        Title = Γραφική, Πλήρωση Πολυγώνων
        Author = {Χρήστος Μάριος Περδίκης},
        Producer = XeLaTeX,
    } 
}

\setmainfont{FreeSerif}


\title{Πλήρωση Πολυγώνων}
\date{Εαρινό Εξάμηνο 2024-2025}
\author{Χρήστος-Μάριος Περδίκης 10075 cperdikis@ece.auth.gr}

\begin{document}
\maketitle
\section{23/5/25 --- Η Αρχή Των Πάντων}
Μπήκα αμέσως στο ψητό, διάβασα εκφώνηση, τα έριξα στον βοηθό και έχω ήδη 
αποτελέσματα. Μου φαίνεται ότι τα αποτελέσματα είναι σκάρτα και garbage.
Ο κώδικας δεν τρέχει ακόμα όταν χρησιμοποιώ τη δική μου polygon_fill 
συνάρτηση, αυτό είναι πρόβλημα. Ακόμα και όταν τρέχει με το fill του
έτοιμου κώδικα τα αποτελέσματα δεν μαρέσουν, αλλά τουλάχιστον πλέον έχω 
debug information. Την επόμενη φορά (αύριο) ας προσπαθήσω να τρέξει ο
κώδικας με το δικό μου polygon_fill και ας δω τι λέει το debug information,
ίσως πάρω καμιά ιδέα από εκεί. Θα ήθελα να μην αργήσω πολύ την εκπόνηση της 
αναφοράς, το πρόγραμμά μ έχει γύρω στις 500 γραμμές συνολικού κώδικα. Επίσης
ακόμα δεν ξέρω αν πρέπει να κάνω gouraud shading ή texture map shading.

\end{document}

