\documentclass{article} 
\usepackage{polyglossia} 
\usepackage{amsmath}
\usepackage{fontspec} 
\usepackage{lipsum} 
\usepackage[margin=1in]{geometry}
\usepackage{graphicx} 
\usepackage{caption} 
\usepackage{subcaption}
\usepackage{hyperref} 
\usepackage{listing}
\hypersetup{% 
    colorlinks=true, linkcolor=blue, filecolor=magenta,      
    urlcolor=cyan, 
    pdfinfo = {%
        Title = Γραφική, Πλήρωση Πολυγώνων
        Author = {Χρήστος Μάριος Περδίκης},
        Producer = XeLaTeX,
    } 
}

\setmainfont{FreeSerif}


\title{Πλήρωση Πολυγώνων}
\date{Εαρινό Εξάμηνο 2024-2025}
\author{Χρήστος-Μάριος Περδίκης 10075 cperdikis@ece.auth.gr}

\begin{document}
\maketitle
\section{23/5/25 --- Η Αρχή Των Πάντων}
Μπήκα αμέσως στο ψητό, διάβασα εκφώνηση, τα έριξα στον βοηθό και έχω ήδη 
αποτελέσματα. Μου φαίνεται ότι τα αποτελέσματα είναι σκάρτα και garbage.
Ο κώδικας δεν τρέχει ακόμα όταν χρησιμοποιώ τη δική μου polygon\_fill 
συνάρτηση, αυτό είναι πρόβλημα. Ακόμα και όταν τρέχει με το fill του
έτοιμου κώδικα τα αποτελέσματα δεν μαρέσουν, αλλά τουλάχιστον πλέον έχω 
debug information. Την επόμενη φορά (αύριο) ας προσπαθήσω να τρέξει ο
κώδικας με το δικό μου polygon\_fill και ας δω τι λέει το debug information,
ίσως πάρω καμιά ιδέα από εκεί. Θα ήθελα να μην αργήσω πολύ την εκπόνηση της 
αναφοράς, το πρόγραμμά μ έχει γύρω στις 500 γραμμές συνολικού κώδικα. Επίσης
ακόμα δεν ξέρω αν πρέπει να κάνω gouraud shading ή texture map shading.

\section{25/5/25 --- Διστακτική Συνέχεια}
Τώρα ξέρω ότι πρέπει να κάνω texture shading. Αυτό περιπλέκει ακόμα περισσότερο
τα πράγματα. Η δική μου texture map shading συνάρτηση δεν δούλευε σωστά, αλλά
πήρα την συνάρτηση από τον Σωτήρη και μπορώ να χρησιμοποιήσω αυτή. Έχω τρία πράγματα 
που μπορώ να κάνω:

\begin{itemize}
    \item να βάλω τη συνάρτηση του Σωτήρη για texture map shading στον κώδικα 
        (αυτό μπορεί να γίνει και στον κώδικα της προηγούμενης εργασίας)
    \item να προσπαθήσω να κάνω τον κώδικα που ήρθε από τα ουράνια να δουλέψει 
        με τον δικό μου κώδικα
    \item να ξεκινήσω την αναφορά
\end{itemize}

Ας αρχίσω με τη συνάρτηση του Σωτήρη και ας κάνω τον κώδικα να δουλέψει,
η αναφορά ας περιμένει μέχρι να ξανατρέξει το πρόγραμμα, ανεξαρτήτου 
αποτελέσματος.

\end{document}

