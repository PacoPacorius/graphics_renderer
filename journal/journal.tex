\documentclass{article} 
\usepackage{polyglossia} 
\usepackage{amsmath}
\usepackage{fontspec} 
\usepackage{lipsum} 
\usepackage[margin=1in]{geometry}
\usepackage{graphicx} 
\usepackage{caption} 
\usepackage{subcaption}
\usepackage{hyperref} 
\usepackage{listing}
\hypersetup{% 
    colorlinks=true, linkcolor=blue, filecolor=magenta,      
    urlcolor=cyan, 
    pdfinfo = {%
        Title = Γραφική, Πλήρωση Πολυγώνων
        Author = {Χρήστος Μάριος Περδίκης},
        Producer = XeLaTeX,
    } 
}

\setmainfont{FreeSerif}
\setlength{\parskip}{1em}
\setlength{\parindent}{0em}


\title{Πλήρωση Πολυγώνων}
\date{Εαρινό Εξάμηνο 2024-2025}
\author{Χρήστος-Μάριος Περδίκης 10075 cperdikis@ece.auth.gr}

\begin{document}
\maketitle
\section{23/5/25 --- Η Αρχή Των Πάντων}
Μπήκα αμέσως στο ψητό, διάβασα εκφώνηση, τα έριξα στον βοηθό και έχω ήδη 
αποτελέσματα. Μου φαίνεται ότι τα αποτελέσματα είναι σκάρτα και garbage.
Ο κώδικας δεν τρέχει ακόμα όταν χρησιμοποιώ τη δική μου polygon\_fill 
συνάρτηση, αυτό είναι πρόβλημα. Ακόμα και όταν τρέχει με το fill του
έτοιμου κώδικα τα αποτελέσματα δεν μαρέσουν, αλλά τουλάχιστον πλέον έχω 
debug information. Την επόμενη φορά (αύριο) ας προσπαθήσω να τρέξει ο
κώδικας με το δικό μου polygon\_fill και ας δω τι λέει το debug information,
ίσως πάρω καμιά ιδέα από εκεί. Θα ήθελα να μην αργήσω πολύ την εκπόνηση της 
αναφοράς, το πρόγραμμά μ έχει γύρω στις 500 γραμμές συνολικού κώδικα. Επίσης
ακόμα δεν ξέρω αν πρέπει να κάνω gouraud shading ή texture map shading.

\section{25/5/25 --- Διστακτική Συνέχεια}
Τώρα ξέρω ότι πρέπει να κάνω texture shading. Αυτό περιπλέκει ακόμα περισσότερο
τα πράγματα. Η δική μου texture map shading συνάρτηση δεν δούλευε σωστά, αλλά
πήρα την συνάρτηση από τον Σωτήρη και μπορώ να χρησιμοποιήσω αυτή. Έχω τρία πράγματα 
που μπορώ να κάνω:

\begin{itemize}
    \item να βάλω τη συνάρτηση του Σωτήρη για texture map shading στον κώδικα 
        (αυτό μπορεί να γίνει και στον κώδικα της προηγούμενης εργασίας)
    \item να προσπαθήσω να κάνω τον κώδικα που ήρθε από τα ουράνια να δουλέψει 
        με τον δικό μου κώδικα
    \item να ξεκινήσω την αναφορά
\end{itemize}

Ας αρχίσω με τη συνάρτηση του Σωτήρη και ας κάνω τον κώδικα να δουλέψει,
η αναφορά ας περιμένει μέχρι να ξανατρέξει το πρόγραμμα, ανεξαρτήτου 
αποτελέσματος.

Το πρόγραμμα τρέχει πάνω από μια ώρα οπότε δεν ξέρω καν αν θα 
τρέξει μέχρι το τέλος ή όχι. Στο desktop μου πέθαναν και οι 
drivers των γραφικών οπότε μάλλον πρέπει να κάνω restart.
Οκ έκανα restart. Τι κάνω τώρα; Ακόμα δεν μπορώ να κάνω το πρόγραμμα
να τρέχει. Μήπως να προσπαθήσω να ξανατρέξω το πρόγραμμα χωρίς την
polygon\_fill του προηγούμενου παραδοτέου; Ναι αυτό θα κάνω.

Το νέο texture map shading που κατέβηκε από τα ουράνια τουλάχιστον
αποδίδει κάποιο αποτέλεσμα. Παρόλα αυτά είναι μπλε για κάποιο λόγο
ενώ πρέπει να είναι πορτοκαλί. Ίσως περιέργειες του color format;
Ακόμα δεν ξέρω αν αυτό που κάνω render είναι σωστό (φαίνεται να 
είναι υπερβολικά μεγάλο για το viewport που έχουμε) και επίσης ακόμα 
κάνει πολύ ώρα για κάθε demo case.

Πλέον είναι πορτοκαλί όπως πρέπει, αλλά είναι ακόμα stretched 
και warped και όχι σωστό.

Ξαναπροσπάθησα με για texture map shading και πλέον το τρίγωνο δεν είναι 
το μισό εκτός της εικόνας. Αλλά το mapping του texture είναι ακόμα 
warped και stretched και κακό. Ίσως να το αγνοήσω για την ώρα όμως,
ή να καλέσω τη συνάρτηση t\_shading του Σώτου μέσα από την render\_object.

\section{27/5/25 --- Αρρώστια}
Λόγω αρρώστιας πιο χαλαρή εναχόληση. Προβλήματα με το rendering
είναι ότι όταν το τρίγωνο βρίσκεται κάτω από τον ορίζοντα, είναι 
πάλι glitched και ότι το texture map shading δεν δουλεύει σωστά.
Επίσης φτιάχνονται αχρείαστα figure και δείχνονται στην οθόνη ενώ 
θέλω απλά να τα φυλάξω στον δίσκο.

Το texture map shading γίνεται σωστά, αλλά δεν είμαι σίγουρος αν το
rendering του τρισδιάστατου αντικειμένου είναι σωστό. Είναι πάλι 
πολύ zoomed in για το case 1, και νομίζω για το case 2 φαίνεται σαν να
μπορούμε να δούμε μέσα του. Νομίζω αύριο θα συνεχίσω να κάνω την αναφορά,
το αποτέλεσμα είναι κάπως εμφανίσημο.
\end{document}

