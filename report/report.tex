\documentclass{article} 
\usepackage{polyglossia} 
\usepackage{amsmath}
\usepackage{fontspec} 
\usepackage{lipsum} 
\usepackage[margin=1in]{geometry}
\usepackage{graphicx} 
\usepackage{caption} 
\usepackage{subcaption}
\usepackage{hyperref} 
\usepackage{listing}
\hypersetup{% 
    colorlinks=true, linkcolor=blue, filecolor=magenta,      
    urlcolor=cyan, 
    pdfinfo = {%
        Title = Γραφική, Πλήρωση Πολυγώνων
        Author = {Χρήστος Μάριος Περδίκης},
        Producer = XeLaTeX,
    } 
}

\setmainfont{FreeSerif}


\title{Πλήρωση Πολυγώνων}
\date{Εαρινό Εξάμηνο 2024-2025}
\author{Χρήστος-Μάριος Περδίκης 10075 cperdikis@ece.auth.gr}

\begin{document}
\maketitle
\emph{INTRO}

\section{Συναρτήσεις Μ/Σ}
\subsection{def translate(t\_vec: np.ndarray) -> np.ndarray}
Σε αυτή τη συνάρτηση αρχικοποιείται ο $4x4$ πίνακας εξόδου xform ως
μοναδιαίος πίνακας. Στην τελευταία στήλη του τοποθετείται το διάνυσμα 
μετατόπισης t\_vec (το κάτω δεξιά στοιχείο του πίνακα μένει ανέπαφο, 
δηλαδή ίσο με 1) και ο $4x4$ πίνακας Μ/Σ μετατόπισης είναι έτοιμος.
\subsection{def rotate(axis: np.ndarray, angle: float, center: np.ndarray) -> np.ndarray}
Η υλοποίηση της συνάρτησης ξεκινάει με την κανονικοποίηση του διανύσματος 
του άξονα περιστροφής axis. Έπειτα υπολογίζεται ο πίνακας $Κ$:

\begin{equation}
    K = \left[\begin{matrix}
        0 & -axis_z & axis_y \\
        axis_z & 0 & -axis_x \\
        -axis_y & axis_x & 0
    \end{matrix}\right]
\end{equation}
ο οποίος χρησιμοποιείται στο matrix notation του τύπου περιστροφής του
Rodrigues ως εξής:

\begin{equation}
    R_{3x3} = I_{3x3} + sin(\theta) * K + (1-cos(\theta)) * K^2
\end{equation}
όπου $\theta$ η γωνία περιστροφής (μεταβλητή εισόδου angle) και $K^2$ το dot 
matrix product του πίνακα $K$. Επειδή ο τύπος του Rodrigues λειτουργεί για
ένα σύστημα συντεταγμένων, θεωρούμε ότι το κέντρο των αξόνων κάθε φορά
είναι το σημείο center. Για αυτό υπολογίζονται οι πίνακες μετατόπισης 
από και προς το σημείο center με τη βοήθεια της παραπάνω συνάρτησης 
translate(). Προκύπτουν οι πίνακες:
\begin{gather*}
    T_{to\_origin} = translate(-center) \\
    T_{back} = translate(center)
\end{gather*}
Ο πίνακας $R$ μετατρέπεται σε ομογενής:
\begin{equation}
    R_{4x4} = \left[\begin{matrix}
        R & 0 \\
        0 & 1
    \end{matrix}\right]
\end{equation}
και τελικά ο  $4x4$ πίνακας εξόδου είναι ο 
\begin{equation}
    xform_{4x4} = T_{back} * R_{4x4} * T_{to\_origin}
\end{equation}

Ουσιαστικά γίνεται πρώτα η μετατροπή συστήματος συντεταγμένων στο 
σύστημα συντεταγμένων του άξονα περιστροφής (με κέντρο το center)
με τον πολλαπλασιασμό με τον πίνακα $T_{to\_origin}$,
γίνεται η περιστροφή με τον πίνακα $R$ και τέλος γίνεται η
αλλαγή στο αρχικό σύστημα συντεταγμένων με τον πίνακα $T_{back}$.

%Prompt:
%I want to make a graphics renderer in python. It is comprised of the following
%functions:
%
%def translate(t_vec: np.ndarray) -> np.ndarray:
%"""
%Create an affine transformation matrix w.r.t. the
%specified translation vector.
%"""
%return xform
%
%t_vec is a 1x3 translation vector and the output is xform,
%a 4x4 affine transformation matrix.
%
%
%def rotate(axis: np.ndarray, angle: float, center: np.ndarray) -> np.ndarray:
%"""
%Create an affine transformation matrix w.r.t. the
%specified rotation.
%"""
%return xform
%
%axis specifies the rotation axis (1x3). angle is the rotation angle in rads (scalar). 
%Rotation counter-clockwise. center is the center point of the rotation axis.
%The ouput xform is the 4x4 affine rotation matrix.
%
%
%def compose(mat1: np.ndarray, mat2: np.ndarray) -> np.ndarray:
%return mat
%
%This function combines two transformation matrices into one.
%
%
%def world2view(pts: np.ndarray, R: np.ndarray, c0: np.ndarray) -> np.ndarray:
%# Implements a world-to-view transform, i.e. transforms the specified
%# points to the coordinate frame of a camera. The camera coordinate frame
%# is specified rotation (w.r.t. the world frame) and its point of reference
%# (w.r.t. to the world frame). This function will make use of the above
%# functions when necessary.
%
%pts is a 3xN matrix with the coords of the points in the original coordinate system.
%R is a rotation matrix of the new coord system in the relation to the original
%coord system. c0 is the point of reference of the new coord system in relation
%to the original.
%
%
%def lookat(eye: np.ndarray, up: np.ndarray, target: np.ndarray) -> Tuple[np.ndarray,
%np.ndarray]:
%# Calculate the camera's view matrix (i.e., its coordinate frame transformation specified
%# by a rotation matrix R, and a translation vector t).
%# :return a tuple containing the rotation matrix R (3 x 3) and a translation vector 
%# t (1 x 3)
%
%eye is the center point of the camera (1x3)
%up is the up vector of the camera (1x3)
%target is the target point the camera points to (1x3)
%
%
%def perspective_project(pts: np.ndarray, focal: float, R: np.ndarray, t: np.ndarray)
%-> Tuple[np.ndarray, np.ndarray]:
%# Project the specified 3d points pts on the image plane, according to a pinhole
%# perspective projection model.
%
%pts is a 3xN matrix with the coords of the input points. focal is focal length
%of the camera. R (3x3) is the rotation matrix w.r.t. the camera coordinate system.
%t is the translation vector (1x3) w.r.t. the camera coordinate system. The function
%returns 2D point coords on the camera plane.
%
%
%def rasterize(pts_2d: np.ndarray, plane_w: int, plane_h: int, res_w: int, res_h: int)
%-> np.ndarray:
%# Rasterize the incoming 2d points from the camera plane to image pixel
%coordinates
%
%The camera plane has a size of plane_h x plane_w. The image has a size of res_h x res_w.
%The beginning of the image is in the bottom left corner and increments going up and to 
%the right. The camera plane's center is in the center of the rectangle plane_h x plane_w.
%
%
%def render_object(v_pos, v_clr, t_pos_idx, plane_h, plane_w, res_h, res_w, focal, eye, up, target) -> np.ndarray:
%# render the specified object from the specified camera.
%
%v_pos (Nx3) are the 3D coordinates of vertices of the object. v_clr (Nx3) is 
%the matrix of the colors corresponding to the vertices. t_pos_idx (Fx3) is the 
%matrix that points to which vertices form the triangles that form the 3D object.
%The i-th line of this matrix contains the vertices (from the matrices
%v_pos and v_clr) that make up a triangle. plane_h and plane_w is the height and 
%the width of the camera plane. res_h and res_w are the image canvas size in pixels.
%focal is the focal length, the distance of the camera plane from the camera center.
%eye is the center of the camera w.r.t. the world coordinate system. up is the 
%up vector of the camera. target is the point the camera points to. The function
%will return the image of the object in a matrix (size res_h x res_w x 3) and 
%will implement the whole pipeline of rendering an object. It will also use the 
%attached files to fill and color the object with the Gouraud method. 
%
%Load the initial data from the hw2.npy file. An object is moving in a circular
%road with the center k_circle_center and radius k_circle_radius with speed
%k_car_velocity. There is a camera next to the object, its relative position to
%the object is k_cam_rel_pos. Create two demos, one for each of the following
%cases. Case 1: the camera is static and the z axis of the camera is always
%in parallel to the speed vector of the car. Case 2: The camera can spin 
%around as the car moves and it always points to the point k_cam_target.
%
%(attach hw2.npy, polygon_fill.py and vec_inter.py)
\end{document}
